\documentclass{article}
\usepackage{latexml}
\usepackage{hyperref}
\usepackage{../sty/latexmldoc}
\usepackage{listings}
% Should the additional keywords be indexed?
\lstdefinestyle{shell}{language=bash,escapechar=@,basicstyle=\ttfamily\small,%
   morekeywords={latexml,latexmlpost,latexmlmath},
   moredelim=[is][\itshape]{\%}{\%}}

\input{releases.tex}
\newcommand{\PDFIcon}{\includegraphics{pdf}}

\title{\LaTeXML\ \emph{A \LaTeX\ to XML/HTML/MathML Converter}}
\lxKeywords{LaTeXML, LaTeX to XML, LaTeX to HTML, LaTeX to MathML, LaTeX to ePub, converter}
%============================================================
\begin{lxNavbar}
\lxRef{top}{\includegraphics{../graphics/latexml}}\\
\includegraphics{../graphics/mascot}\\
\lxContextTOC\\
% Direct link to manual from navbar
\vspace{1cm}
\URL[\hspace{4em}\LARGE The\ Manual]{./manual/}
\end{lxNavbar}
%============================================================
\lxDocumentID{ltxsite}
\begin{document}
\label{top}
\maketitle

%============================================================
\emph{Now available}:  \htmlref{\LaTeXML\ \CurrentVersion}{get}!

In the process of developing the
\href{http://dlmf.nist.gov/}{Digital Library of Mathematical Functions},
we needed a means of transforming
the \LaTeX\ sources of our material into XML which would be used
for further manipulations, rearrangements and construction of the web site.
In particular, a true `Digital Library' should focus on the \emph{semantics}
of the material, and so we should convert the mathematical material into both
content and presentation MathML.
At the time, we found no software suitable to our needs, so we began
development of \LaTeXML\ in-house.  

In brief, \texttt{latexml} is a program, written in Perl, that attempts to
faithfully mimic \TeX's behavior, but produces XML instead of dvi.
The document model of the target XML makes explicit the model implied
by \LaTeX.
The processing and model are both extensible; you can define
the mapping between \TeX\ constructs and the XML fragments to be created.
A postprocessor, \texttt{latexmlpost} converts this
XML into other formats such as HTML or XHTML, with options
to convert the math into MathML (currently only presentation) or images.

\emph{Caveats}: It isn't finished, there are gaps in the coverage,
particularly in missing implementations of the many useful \LaTeX\ packages.
But is beginning to stabilize and interested parties
are invited to try it out, give feedback and even to help out.


%============================================================
\section{Examples}\label{examples}\index{examples}
At the moment, the best example of \LaTeXML's output is 
the \href{http://dlmf.nist.gov/}{DLMF} itself.
There is, of course, a fair amount of insider, special case,
code, but it shows what can be done.

Some highlights:
\begin{description}
\item[\href{examples/tabular/tabular.html}{LaTeX tabular}]
    from the \LaTeX\ manual, p.205.
    (\href{examples/tabular/tabular.tex}{\TeX},
     \href{examples/tabular/tabular.pdf}{\PDFIcon})
\item[\url{http://latexml.mathweb.org/editor}] an online editor/showcase
  of things that \LaTeXML\ can do.
\item[\url{http://arxmliv.kwarc.info}] An experiment processing
  the entire \url{http://arXiv.org}.
\end{description}
And, of course
\begin{description}
\item[\href{http://dlmf.nist.gov/}{DLMF}]
   The Digital Library of Mathematical Functions was the
   primary instigator for this project.
\item[\href{manual/}{\LaTeXML\ Manual}]
   The \LaTeXML\ User's manual (\href{manual.pdf}{\PDFIcon}).
\item[And these pages] were produced using \LaTeXML, as well.
\end{description}

%============================================================
\section{Get \LaTeXML}\label{get}\index{get}
\paragraph{Current Release:}\label{download.current}
The current release is \CurrentVersion. (see the \href{Changes}{Change Log}).
\emph{Prebuilt packages will be updated shortly.}

There are several ways to install \LaTeXML, depending on your OS platform,
whether you want a bleeding-edge version from GitHub.  Since \LaTeXML\ depends
on, or can use if available, several other Perl Modules and external programs,
consult \ref{prerequisites} before deciding.
\begin{itemize}
\item \emph{Normally}, it is preferable to install a platform-specific, prebuilt release,
  if available,  which will install all prerequisites,  usually including the optional ones.

\item You may install from \htmlref{CPAN}{install.cpan} which will install most prerequisites,
  although you may prefer to pre-install prerequisites using the platform-specific approach
   and should pre-install any optional prerequisites that you want.

\item You may download and install the \htmlref{tarball}{install.tarball},
   but you will need to pre-install prerequisites (including any optional ones)
   using \htmlref{CPAN}{install.cpan.prereq} or a platform-specific approach.

\item If you want to use the development version with the latest patches and improvements,
  you may fetch the source from \htmlref{GitHub}{install.github}.
  Again you will need to pre-install prerequisites (including optional ones)
   using \htmlref{CPAN}{install.cpan.prereq} or a platform-specific approach.
  If you are lacking root rights on the machine you want to install on, this can
   also be achieved conveniently using \htmlref{CPANM}{install.cpanm}. 
\end{itemize}

\par\noindent
\begin{tabular}{l|l|l}
Platform & To install prebuilt & To install prerequisites \\\hline
RPM-based Linux
   & see \htmlref{RPM prebuit}{install.rpm}
   & see \htmlref{RPM prerequisites}{install.rpm.prereq} \\
Debian-based Linux
   & see \htmlref{Debian prebuilt}{install.deb}
   & see \htmlref{Debian prerequisites}{install.deb.prereq}\\
Macintosh OS w/\href{http://www.macports.org}{MacPorts}
   & see \htmlref{MacPorts prebuilt}{install.mac}
   & see \htmlref{MacPorts prerequisites}{install.mac.prereq}\\
Windows w/\href{http://strawberryperl.com}{Strawberry Perl}
   & see \htmlref{Windows CPAN}{install.windows}
   & see \htmlref{Windows prerequisites}{install.windows.prereq}  \\
Other
   & see \htmlref{CPAN}{install.cpan} or \htmlref{GitHub}{install.github}
   & see \htmlref{CPAN prerequisites}{install.cpan.prereq} \\
\end{tabular}
Note that there is \emph{no} implied endorsement of any of these systems.

\paragraph{Prerequisites}\label{prerequisites}
\LaTeXML\ requires several Perl modules to do it's job.  Most
are automatically installed by the platform-specific installation or CPAN.
However, CPAN will \emph{not} install the required C libraries needed for
\texttt{XML::LibXML}, and \texttt{XML::LibXSLT}.
If \texttt{libxml2} and \texttt{libxslt} are are not already installed,
follow the instructions at \href{http://www.xmlsoft.org}{XMLSoft} to
download and install the most recent versions of \texttt{libxml2} and \texttt{libxslt}.
Note that Strawberry Perl, on Windows, already includes these libraries
(but ActiveState does not).

\paragraph{Optional Prerequisites}
The following packages are optional because they are sometimes difficult
to find or install, or in order to allow for minimal installs in unusual
circumstances.  Most users should consider them as required and install
them if at all possible.
\begin{description}
\item[\TeX] Virtually all users of \LaTeXML\ will want to install \TeX.  \LaTeXML\ 
\emph{should} find whatever \TeX-installation you have, and will
use \TeX's style files directly in some cases, providing broader coverage,
particularly for the more complex styles like \texttt{babel} and \texttt{tikz}.
Moreover, if \TeX\ is present when \LaTeXML\ is being installed,
\LaTeXML\ will install a couple of its own style files that can be used
with regular \TeX, or \LaTeX\ runs;
So if you are going to install \TeX, install it first!

\item[Image::Magick] provides a handy library of image manipulation routines.
When they are present \LaTeXML\ is able to carry out more image processing,
such as transformations by the \texttt{graphicx} package, and conversion of math to images;
otherwise, some such operations will not be supported.

Please do \emph{not} try to install \texttt{Image::Magick} from CPAN, however:
the module there seldom matches the underlying \texttt{ImageMagick} library.
It is recommended to install the Perl binding for \texttt{Image::Magick} from
the same source as the library was obtained, either from your system's repository,
or from the \href{http://www.imagemagick.org/}{ImageMagick} site, itself.
In the latter case, follow the instructions at \href{http://www.imagemagick.org/}{ImageMagick}
to download and install the latest version of ImageMagick being sure to enable
and build the Perl binding along with it.

\item[Graphics::Magick] is an \emph{alternative} to \texttt{Image::Magick} that \LaTeXML\ will
use if is found on the system; it may (or may not ) be easier to install, although it
is less widely available.

\item[UUID::Tiny] generates unique identifiers that can be used to make better ePub documents
  (it can be installed using \htmlref{CPAN}{install.cpan.prereq}).

\item[perl-doc]\label{perl-doc}
 On \emph{some} compact distributions the perl documentation modules
 are not installed by default (eg.~debian minimal). These modules help generate
 readable command-line documentation for the \LaTeXML\ tools. Thus you \emph{may}
 want to install an extra package (\texttt{perl-doc} on debian minimal) to enable this feature. 

\end{description}
\emph{Note to packagers:} If you are preparing a compiled installation package (such as rpm or deb) for
\LaTeXML, and the above packages are easily installable in your distribution,
you probably should include them as dependencies of \LaTeXML.


%\subsection[OS-Specific Notes]{Operating System Specific Notes}\label{install.osnotes}
%With \emph{no} implied endorsement of any of these systems.

\subsection[RPM-based systems]{RPM-based systems}\label{install.rpm}
\index{rpm}\index{Fedora}\index{Redhat}\index{Centos}
For Fedora and Enterprise Linux based distributions
(Redhat Enterprise Linux, CentOS, Scientific Linux\ldots) and similar,
most software is obtained and installed via the yum repository.

\paragraph{Installing prebuilt}\\
The following commands will install \LaTeXML, including its required
and optional prerequisites:
\begin{itemize}
\item Download \CurrentFedora\ or \CurrentRedHat\, % (but enable the elp repository),
as appropriate.
\item Install \LaTeXML, and its prerequisites, using the command:
\begin{lstlisting}[style=shell]
yum --nogpgcheck localinstall LaTeXML-@\CurrentVersion@-*.noarch.rpm
\end{lstlisting}
\end{itemize}

\paragraph{Installing prerequisites}\label{install.rpm.prereq}
The prerequisites (including the optional ones) can be installed
by running this command as root: 
\begin{lstlisting}[style=shell]
yum install \
  perl-Archive-Zip perl-DB_File perl-File-Which \
  perl-Getopt-Long perl-Image-Size perl-IO-String perl-JSON-XS \
  perl-libwww-perl perl-Parse-RecDescent perl-Pod-Parser \
  perl-Text-Unidecode perl-Test-Simple perl-Time-HiRes perl-URI \
  perl-XML-LibXML perl-XML-LibXSLT \
  perl-UUID-Tiny texlive ImageMagick ImageMagick-perl
\end{lstlisting}
Continue by installing \LaTeXML\ from
\htmlref{tarball}{install.tarball}, \htmlref{CPAN}{install.cpan}
or \htmlref{GitHub}{install.github}, as desired.

\subsection{Debian-based systems}\label{install.deb}\index{Debian}
For Debian-based systems (including Ubuntu), the deb repositories
are generally used for software installation.
Thanks to Peter Ralph and Atsuhito Kohda, \LaTeXML\ is available from Debian's
\emph{unstable} repositories (the version in the stable repositories may be quite old).

\paragraph{Installing prebuilt}
The following command will install \LaTeXML, including its required
and optional prerequisites:
\begin{lstlisting}[style=shell]
sudo apt-get install latexml
\end{lstlisting}

\paragraph{Installing prerequisites}\label{install.deb.prereq}
The prerequisites (including optional ones) can be installed
by running this command: 
\begin{lstlisting}[style=shell]
sudo apt-get install   \
  libarchive-zip-perl libfile-which-perl libimage-size-perl  \
  libio-string-perl libjson-xs-perl libtext-unidecode-perl \
  libparse-recdescent-perl liburi-perl libuuid-tiny-perl libwww-perl \
  libxml2 libxml-libxml-perl libxslt1.1 libxml-libxslt-perl  \
  texlive-latex-base imagemagick libimage-magick-perl
\end{lstlisting}
Continue by installing \LaTeXML\ from
\htmlref{tarball}{install.tarball}, \htmlref{CPAN}{install.cpan}
or \htmlref{GitHub}{install.github}, as desired.

See note \ref{perl-doc} about optional installation of \texttt{perl-doc}.

\subsection{Archlinux \& friends}\label{install.arch}\index{Arch}
For Archlinux and derivatives, it is most convenient to install from sources via \htmlref{CPANM}{install.cpanm}. 
Nonetheless, a package can be found in the \href{https://aur.archlinux.org/}{Archlinux User Repository}.
Furthermore, most dependencies can be found in the official repositories.

\paragraph{Installing from AUR}
To install latexml from the user repositories, install the \href{https://aur.archlinux.org/packages/perl-latexml/}{perl-latexml} package. 

\paragraph{Installing dependencies}
To install the dependencies, use the following pacman command:
\begin{lstlisting}[style=shell]
sudo pacman -S db imagemagick perl perl-algorithm-diff \
   perl-archive-zip perl-file-which perl-image-size \
   perl-io-string perl-libwww perl-json-xs \
   perl-parse-recdescent perl-xml-libxml perl-xml-libxslt \
   texlive-core
\end{lstlisting}

Additionally, install the \href{https://aur.archlinux.org/packages/perl-text-unidecode/}{perl-text-unidecode}
AUR package. 



\subsection{MacOS}\label{install.mac}\index{Apple Macintosh}
For Apple Macintosh systems, the  \href{http://www.macports.org}{MacPorts}
repository is the most convenient way to install \LaTeXML;
thanks to devens, Mojca, Sean and Andrew Fernandes.
Download and install MacPorts from that site.

\paragraph{Installing prebuilt}
Since some users prefer \href{http://tug.org/mactex/}{MacTeX} or another
\TeX\ system over MacPort's \texttt{texlive}, and since \texttt{texlive} is quite large,
the \emph{default} \LaTeXML\ installation does \emph{not} include a dependency
on a \TeX-installation, although it installs \LaTeXML's own style files where \texttt{texlive}
would expect them, if you do choose to install it.
Note that \LaTeXML\ will use, for its own purposes, the style files from whatever
\TeX\ system, if any, it finds at runtime.

\subparagraph{TeXLive or Other:} To install \LaTeXML\ to work with MacPort's texlive,
or without any installed \TeX, use the command:
\begin{lstlisting}[style=shell]
sudo port install LaTeXML
\end{lstlisting}

\subparagraph{MacTeX:}  To install \LaTeXML, along with its prerequisites,
and installing style files where MacTeX (assumed already installed) will find them,
use the command:
\begin{lstlisting}[style=shell]
sudo port install LaTeXML +mactex
\end{lstlisting}

\paragraph{Installing prerequisites}\label{install.mac.prereq}
The prerequisites (including optional ones except \TeX) can be installed by running
this command as root: 
\begin{lstlisting}[style=shell]
sudo port install \
  p5-archive-zip p5-file-which p5-getopt-long p5-image-size \
  p5-io-string p5-json-xs p5-text-unidecode p5-libwww-perl \
  p5-parse-recdescent p5-time-hires p5-uri p5-xml-libxml \
  p5-xml-libxslt p5-perlmagick
\end{lstlisting}
Additionally, either add \texttt{texlive} to the above list,
or install \href{http://tug.org/mactex/}{MacTeX}.
Continue by installing \LaTeXML\ from
\htmlref{tarball}{install.tarball}, \htmlref{CPAN}{install.cpan}
or \htmlref{GitHub}{install.github}, as desired.

\subsection{Windows}\index{windows}
These installation instructions assume you will use
\href{http://strawberryperl.com}{Strawberry Perl},
which comes with \emph{many} of our prerequisites pre-installed,
and provides other needed commands (\texttt{perl}, \texttt{cpan}, \texttt{dmake}).

\paragraph{Installing prebuilt}\label{install.windows}
There is currently no prebuilt \LaTeXML\ for Windows,
but it should install cleanly under \href{http://strawberryperl.com}{Strawberry Perl}'s CPAN.
Install the \TeX-system of your choice (if desired),
ImageMagick (see \ref{install.windows.imagemagick})
and then install LaTeXML using:
\begin{lstlisting}[style=shell]
cpan LaTeXML
\end{lstlisting}

Installing the optional package \texttt{Image::Magick} on Windows seems to be problematic,
so we have omitted it from these instructions.
You may want to try \href{ImageMagick}, but
you're on your own, there!  You may  have better luck with \texttt{Graphics::Magick}.

\paragraph{Installing prerequisites}\label{install.windows.prereq}\\
Install \href{http://strawberryperl.com}{Strawberry Perl}, and
the \TeX-system of your choice (if desired), 
ImageMagick (if you want to try; see \ref{install.windows.imagemagick}),
and then install the additional prerequisites as
\begin{lstlisting}[style=shell]
cpan Image::Size Parse::RecDescent UUID::Tiny
\end{lstlisting}
Continue by installing \LaTeXML\ from
\htmlref{tarball}{install.tarball}, \htmlref{CPAN}{install.cpan}
or \htmlref{GitHub}{install.github}, as desired,
but note that the \texttt{dmake} command should be used in place of \texttt{make}.

\paragraph{Installing ImageMagick under Windows}\label{install.windows.imagemagick}
In principle, you should be able to install the binary from ImageMagick
and then install the Perl binding using CPAN; unfortunately, it seems that the
CPAN version seldom matches the binary or fails for other reasons.  What
should work the following:

Download and install the main ImageMagick binary appropriate for your Windows system
from \href{http://imagemagick.org/script/binary-releases.php#windows}{ImageMagick}.
Then fetch the \texttt{PerlMagick} tarball \emph{with the same version} from
\href{http://imagemagick.com/download/perl/}{ImageMagick/perl}.
Use the following commands to compile and install the PerlMagick,
with X.XX being the version you downloaded:
\begin{lstlisting}[style=shell]
tar -zxvf PerlMagick-X.XX.tar.gz
cd PerlMagick-X.XX
perl Makefile.PL
dmake
dmake test
dmake install
\end{lstlisting}

\subsection{CPAN installation}\label{install.cpan}
The following command will install \LaTeXML\ and its Perl prerequisites,
but you may need to pre-install \texttt{libxml2} and  \texttt{libxslt} (See \ref{prerequisites}).
Pre-install \TeX\ and any optional Perl modules, if desired.
\begin{lstlisting}[style=shell]
cpan LaTeXML
\end{lstlisting}

\paragraph{Installing prerequisites}\label{install.cpan.prereq}
The following command will install the Perl prerequisites (including optional)
for \LaTeXML, but you may need to pre-install \texttt{libxml2} and  \texttt{libxslt} (See \ref{prerequisites}).
\begin{lstlisting}[style=shell]
cpan Archive::Zip DB_File File::Which Getopt::Long Image::Size \
  IO::String JSON::XS LWP MIME::Base64 Parse::RecDescent \
  Pod::Parser Text::Unidecode Test::More URI \
  XML::LibXML XML::LibXSLT UUID::Tiny
\end{lstlisting}
You may still want to install \TeX\ and \texttt{Image::Magick}
using other means.

\subsection{CPANM Installation}\label{install.cpanm}
On certain linux machines, you may not want to install \LaTeXML\ (or its dependencies)
system-wide, or you may simply lack the required root rights to do so. 
In such a case, it is convenient to install the development version and 
dependencies into the home directory using a tool called 
\href{ https://github.com/miyagawa/cpanminus}{cpanminus}.

\paragraph{Bootstrapping cpanminus} Configuring and setting up \texttt{cpanminus} can be achieved using the following commands
\begin{lstlisting}[style=shell]
    # Download and install cpanminus
    curl -L http://cpanmin.us | perl - App::cpanminus
    
    # Setup a directory in ~/perl5 to contain all perl dependencies
    ~/perl5/bin/cpanm --local-lib=~/perl5 local::lib && eval $(perl -I ~/perl5/lib/perl5/ -Mlocal::lib)
\end{lstlisting}

\paragraph{Installing} Afterwards installing \LaTeXML\ and dependencies can be achieved with the 
simple command:
\begin{lstlisting}[style=shell]
    cpanm git://github.com/brucemiller/LaTeXML.git
\end{lstlisting}
This automatically fetches the latest version from GitHub and installs 
missing dependencies. 

\subsection{Installing tarball}\label{install.tarball}
\paragraph{Source tarball}\label{source.tarball}
To download the tar file containing the source:
\begin{itemize}
\item Download \CurrentTarball
\item Unpack using
\begin{lstlisting}[style=shell]
tar zxvf LaTeXML-@\CurrentVersion@.tar.gz
\end{lstlisting}
\end{itemize}

\paragraph{Building}\label{build.source}
Whether you've downloaded using the tar file or from git,
you build the system using the standard Perl procedure
(On Windows systems, use \texttt{dmake} instead of \texttt{make}):
\begin{lstlisting}[style=shell]
cd LaTeXML  [@or@  LaTeXML-@\CurrentVersion@ @or@  LaTeXML-master]
perl Makefile.PL
make
make test
\end{lstlisting}
The last step runs the system through some basic tests,
which should complete without error (although some tests may be `skipped'
under certain circumstances).


\emph{Note:} You can specify nonstandard place to install files
--- possibly avoiding the need to install as root! ---
by modifying the Makefile creating command above to
\begin{lstlisting}[style=shell]
perl Makefile.PL PREFIX=%perldir% TEXMF=%texdir%
\end{lstlisting}
where \emph{perldir} is where you want the perl related files to go and
\emph{texdir} is where you want the \TeX\ style files to go.
(See \texttt{perl perlmodinstall} for more details and options.)

\paragraph{Installing}\label{install.source}
After building, you will finally want to install \LaTeXML\ where the OS can find the files.  
You'll typically need to be root:
\begin{lstlisting}[style=shell]
su
make install
\end{lstlisting}
or perhaps
\begin{lstlisting}[style=shell]
sudo make install
\end{lstlisting}
(On Windows, use \texttt{dmake}).

\subsection{Installing from GitHub}\label{install.github}
The development version of \LaTeXML\ is available on  \href{https://github.com}{GitHub}.
This will include the latest patches and enhancements between official releases,
but may of course introduce new bugs or incompatibilities with little notice
(you may wish to subscribe to the \LaTeXML\ \htmlref{mailing list}{contact.list}).
You can also browse the current code at GitHub, as well as file bug reports and enhancement requests.

Fetch the development version using the following command:
\begin{lstlisting}[style=shell]
git clone https://github.com/brucemiller/LaTeXML.git
\end{lstlisting}
Continue with building and installing as described in \ref{build.source}.

Keep up-to-date by occasionally running the command:
\begin{lstlisting}[style=shell]
git pull
\end{lstlisting}
in the source directory, and then repeating the building and install commands.

You can avoid directly using \texttt{git}, but still get the development sources
by downloading
\href{https://github.com/brucemiller/LaTeXML/archive/master.zip}{LaTeXML-master.zip}
and running
\begin{lstlisting}[style=shell]
unzip LaTeXML-master.zip
\end{lstlisting}

\subsection{Archived Releases:}\label{archive}
\AllReleases.

%============================================================
\section{Documentation}\label{docs}
If you're lucky, all that should be needed to convert
a \TeX\ file, \textit{mydoc}\texttt{.tex} to XML, and
then to HTML would be:
\begin{lstlisting}[style=shell]
   latexml --dest=%mydoc%.xml %mydoc%
   latexmlpost --dest=%somewhere/mydoc%.html %mydoc%.xml
\end{lstlisting}
This will carry out a default transformation into HTML5,
which represents mathematics using MathML.  Using an
extension of xhtml would generate XHTML including MathML.
Adding the option \verb|--format=html4| to the call to \verb|latexmlpost|
will generate HTML (version 4), using images to represent the math.

If you're not so lucky, or want to get fancy, well \ldots dig deeper:
\begin{description}
\item[\href{manual/}{LaTeXML Manual}] (\href{manual.pdf}{\PDFIcon}).
\item[\href{manual/commands/latexml.html}{\texttt{latexml}}]
    describes the \texttt{latexml} command.
\item[\href{manual/commands/latexmlpost.html}{\texttt{latexmlpost} command}]
   describes the \texttt{latexmlpost} command for postprocessing.
\end{description}

% Possibly, eventually, want to expose:
%   http://www.mathweb.org/wiki/????
% But, it doesn't have anything in it yet.

%============================================================
\section{Contacts \& Support}\label{contact}

\paragraph{Mailing List}\label{contact.list}
There is a low-volume mailing list for questions, support and comments.
See \href{http://lists.informatik.uni-erlangen.de/mailman/listinfo/latexml}{\texttt{latexml-project}} for subscription information.

\paragraph{Bug-Tracker}\label{contact.git}
We are using the git repository hosted at
\href{https://github.com/brucemiller/LaTeXML/}{https://github.com/brucemiller/LaTeXML/}.
You can browse the code, the latest changes, and check-out the current code from
there (see \ref{get}).  There is also an Issues database where you can
file bug reports or feature requests.

%  There is a Trac bug-tracking system for reporting bugs, or checking the
%  status of previously reported bugs at
%  \href{https://trac.mathweb.org/LaTeXML/}{Bug-Tracker}.

% To report bugs, please:
% \begin{itemize}
% \item \href{http://trac.mathweb.org/register/register}{Register} a Trac account
%   (preferably give an email so that you'll get notifications about activity regarding the bug).
% \item \href{http://trac.mathweb.org/LaTeXML/newticket}{Create a ticket}
% \end{itemize} 

\paragraph{Thanks} to our friends at
the \href{http://kwarc.info}{KWARC Research Group}
for hosting the mailing list, the original Trac system and svn repository,
as well as general moral support.

\paragraph{Author} \href{mailto:bruce.miller@nist.gov}{Bruce Miller}.
%============================================================
\section{License \& Notices}\label{notices}

\paragraph{License}
  This software was developed at the National Institute of Standards and
Technology by employees of the Federal Government in the course of their
official duties. Pursuant to title 17 Section 105 of the United States
Code, this software is not subject to copyright protection in the U.S.
and is in the \textbf{public domain}.

To the extent that any copyright protections may be considered to be held
by the authors of this sofware in some jurisdiction outside the United
States, the authors hereby waive those copyright protections and dedicate
the software to the public domain. Thus, this license may be considered equivalent to
\href{http://creativecommons.org/about/cc0}{Creative Commons 0: "No Rights Reserved"}.

Note that, according to
\href{http://www.gnu.org/licences/license-list.html#PublicDomain}{Gnu.org},
public domain is compatible with GPL.

We would appreciate acknowledgement if the software is used.

\paragraph{Contributor Notice}
  Contributions of software patches and enhancements to this project
are welcome; such contributions are assumed to be under the same terms
as the software itself.  Specifically, if you contribute code, documention,
text samples or any other material, you are asserting and acknowledging that:
you are the copyright holder of the material or that it is in the public domain;
it does not contain any patented material; and that you waive any copyright
protections and dedicate the material to the public domain.

\paragraph{Disclaimer}
  \LaTeXML\ is an experimental system provided by NIST as a public service.

The software is expressly provided "AS IS." NIST makes NO warranty of
any kind, express, implied, in fact or arising by operation of law,
including, without limitation, the implied warranty of
merchantability, fitness for a particular purpose, non-infringement
and data accuracy. nist neither represents nor warrants that the
operation of the software will be uninterrupted or error-free, or that
any defects will be corrected. NIST does not warrant or make any
representations regarding the use of the software or the results
thereof, including but not limited to the correctness, accuracy,
reliability, or usefulness of the software.

You are solely responsible for determining the appropriateness of
using and distributing the software and you assume all risks
associated with its use, including but not limited to the risks and
costs of program errors, compliance with applicable laws, damage to or
loss of data, programs or equipment, and the unavailability or
interruption of operation. This software is not intended to be used in
any situation where a failure could cause risk of injury or damage to
property.

\paragraph{Privacy Notice}
We adhere to \href{http://www.nist.gov/public_affairs/privacy.cfm}{NIST's Privacy, Security and Accessibility Policy}.
%============================================================

\end{document}
